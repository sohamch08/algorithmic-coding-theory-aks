\chapter{Mathematics}
\section{Divisors}
\section{Reimann-Roch Spaces}
\begin{definition}[Reimann-Roch Spaces]
	For any divisor $\mcD\in \tilde{\mfD}$ $$\mcL(\mcD)=\{f\in \bbF(\mcX)^*\mid (f)+\mcD\succcurlyeq 0 \}\cup \{0\}$$The dimension of $\mcL(\mcD)$ over $\bbF$ is denoted by $l(\mcD)$
\end{definition}
\begin{theorem}
	\begin{enumerate}[label=(\roman*)]\label{rrspacedim}
		\item If $\deg(\mcD)<0$ then $l(\mcD)=0$
		\item $l(\mcD)\leq 1+\deg(\mcD)$
	\end{enumerate}
\end{theorem}
\begin{theorem}\label{0divrrspacedim}
	$\mcL(0)=\bbF$. Hence $l(0)=1$
\end{theorem}

\section{Differentials}

\section{Reimann-Roch Theorem}

\begin{theorem}[Reimann-Roch Theorem]
	$\mcD$ is a divisor on a smooth projective curve with genus $g$. Then for any canonical divisor $W$ $$l(\mcD) - l(W-\mcD)= \deg(\mcD)- (g-1)$$
\end{theorem}
\begin{corollary}\label{candivdeg}
	For any canonical divisor $W$,  $\deg(W)=2g-2$
\end{corollary}
\begin{proof}
	Take $\mcD=W$. Then $l(W-\mcD)=l(0)=1$ by \thmref{0divrrspacedim}. So we have $$l(W)-1=\deg(W)-(g-1)$$. By definition $l(W)=g$. Hence we have $g-1=\deg(W)-(g-1)\iff \deg(W)=2g-2$.
\end{proof}
With the help of this corollary we can finally focus on the divisors which we will actually use to define codes. The following corollary gives the dimension of the Reimann-Roch Spaces of divisors with degree more than $2g-2$.
\begin{corollary}\label{divdimdeg}
	Let $\mcD$ be a divisor on a smooth projective curve of genus $g$ and let $\deg(\mcD)>2g-2$. Then $$l(\mcD)=\deg(D)-(g-1)$$
\end{corollary}
\begin{proof}
	We have $\deg(W-\mcD)=\deg(W)-\deg(\mcD)$. Now by \corref{candivdeg} $\deg(W-\mcD)<0$. So0 $l(W-\mcD)=0$ by \thmref{rrspacedim} part (ii). So We have $l(D)=\deg(D)-(g-1)$.
\end{proof}
\section{Index of speciality}
\begin{definition}[Index of speciality]
	Let $\mcD$ be a divisor on a curve $\mcX$. We define $$\Om(\mcD)=\{\om\in \Om(\mcX)\mid (w)-D\succcurlyeq 0\}$$ and we denote the dimension of $\Om(\mcD)$ over $\bbF$ by $\delta(\mcD)$ called the index of speciality of $\mcD$.
\end{definition}
\begin{theorem}
	$\delta(\mcD)=l(W-\mcD)$
\end{theorem}
\begin{proof}
	If $W=(\om)$. Define the linear map $\varphi:\mcL(W-\mcD)\to \Om(\mcD)$ by $\varphi(f)=f\om$. $$f\in \mcL(W-\mcD)\implies (f)+W-\mcD\succcurlyeq 0 \iff (f)+(\om)-\mcD\succcurlyeq \iff (f\om)-\mcD\succcurlyeq0\iff f\in \Om(\mcD)$$Hence $\varphi$ is an isomorphism. Therefore $\delta(\mcD)=l(W-\mcD)$
\end{proof}